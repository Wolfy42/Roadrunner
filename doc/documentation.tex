\documentclass[11pt,a4paper]{article}

% für \hyphenation mit Umlauten
\usepackage[T1]{fontenc}
\usepackage[utf8]{inputenc}
\usepackage[ngerman,english]{babel}

% Times-Roman-Schrift (auch für mathematische Formeln)
\usepackage{mathptmx} 

% comments
\usepackage{verbatim} 

% Zum Setzen von URLs
\usepackage{color}
\usepackage{alltt}
\definecolor{darkred}{rgb}{.25,0,0}
\definecolor{darkgreen}{rgb}{0,.2,0}
\definecolor{darkmagenta}{rgb}{.2,0,.2}
\definecolor{darkcyan}{rgb}{0,.15,.15}
\definecolor{LightButter}{rgb}{0.98,0.91,0.31}
\definecolor{LightOrange}{rgb}{0.98,0.68,0.24}
\definecolor{LightChocolate}{rgb}{0.91,0.72,0.43}
\definecolor{LightChameleon}{rgb}{0.54,0.88,0.20}
\definecolor{LightSkyBlue}{rgb}{0.45,0.62,0.81}
\definecolor{LightPlum}{rgb}{0.68,0.50,0.66}
\definecolor{LightScarletRed}{rgb}{0.93,0.16,0.16}
\definecolor{Butter}{rgb}{0.93,0.86,0.25}
\definecolor{Orange}{rgb}{0.96,0.47,0.00}
\definecolor{Chocolate}{rgb}{0.75,0.49,0.07}
\definecolor{Chameleon}{rgb}{0.45,0.82,0.09}
\definecolor{SkyBlue}{rgb}{0.20,0.39,0.64}
\definecolor{Plum}{rgb}{0.46,0.31,0.48}
\definecolor{ScarletRed}{rgb}{0.80,0.00,0.00}
\definecolor{DarkButter}{rgb}{0.77,0.62,0.00}
\definecolor{DarkOrange}{rgb}{0.80,0.36,0.00}
\definecolor{DarkChocolate}{rgb}{0.56,0.35,0.01}
\definecolor{DarkChameleon}{rgb}{0.30,0.60,0.02}
\definecolor{DarkSkyBlue}{rgb}{0.12,0.29,0.53}
\definecolor{DarkPlum}{rgb}{0.36,0.21,0.40}
\definecolor{DarkScarletRed}{rgb}{0.64,0.00,0.00}
\definecolor{Aluminium1}{rgb}{0.93,0.93,0.92}
\definecolor{Aluminium2}{rgb}{0.82,0.84,0.81}
\definecolor{Aluminium3}{rgb}{0.73,0.74,0.71}
\definecolor{Aluminium4}{rgb}{0.53,0.54,0.52}
\definecolor{Aluminium5}{rgb}{0.33,0.34,0.32}
\definecolor{Aluminium6}{rgb}{0.18,0.20,0.21}

\usepackage[plainpages=false,bookmarks=true,bookmarksopen=true,colorlinks=true,
  linkcolor=darkred,citecolor=darkgreen,filecolor=darkmagenta,
  menucolor=darkred,urlcolor=darkcyan]{hyperref}

% Zeilenabstand
\renewcommand{\baselinestretch}{1.5}

% anhang
\usepackage[toc,page]{appendix}

% pdflatex: Bilder in den Formaten .jpeg, .png und .pdf
% latex: Bilder im .eps-Format
\usepackage{graphicx}

\usepackage{fancyhdr} % Positionierung der Seitenzahlen
\fancyhead[R]{}
\fancyfoot[R]{\Roman{page}}

\renewcommand{\headrulewidth}{0pt}

 % behebt headheight Warning
\setlength{\headheight}{13.6pt}

% Korrektes Format für Nummerierung von Abbildungen (figure) und
% Tabellen (table): <Kapitelnummer>.<Abbildungsnummer>
\makeatletter
\@addtoreset{figure}{section}
\renewcommand{\thefigure}{\thesection.\arabic{figure}}
\@addtoreset{table}{section}
\renewcommand{\thetable}{\thesection.\arabic{table}}
\makeatother

% Listings für Sourcecode
\usepackage{listings}
  \usepackage{courier}
 \lstset{
        basicstyle=\footnotesize\ttfamily, % Standardschrift
        numbers=left,               % Ort der Zeilennummern
        numberstyle=\tiny,          % Stil der Zeilennummern
        %stepnumber=2,               % Abstand zwischen den Zeilennummern
        numbersep=5pt,              % Abstand der Nummern zum Text
        tabsize=2,                  % Groesse von Tabs
        extendedchars=true,         %
        breaklines=true,            % Zeilen werden Umgebrochen
        keywordstyle=\color{red},
        frame=b,         
        keywordstyle=[1]{\color{DarkSkyBlue}},    % Stil der Keywords
        keywordstyle=[2]{\color{DarkScarletRed}},    %
        keywordstyle=[3]{\bfseries},    %
        keywordstyle=[4]{\color{DarkPlum}},    %
        keywordstyle=[5]{\color{SkyBlue}},    %
		stringstyle={\color{Chocolate}},
        showspaces=false,           % Leerzeichen anzeigen ?
        showtabs=false,             % Tabs anzeigen ?
        xleftmargin=17pt,
        framexleftmargin=17pt,
        framexrightmargin=5pt,
        framexbottommargin=4pt,
        backgroundcolor=\color{Aluminium1},
        showstringspaces=false,      % Leerzeichen in Strings anzeigen ?
		%language=php
		morekeywords=[1]{Interface,return,static,function}
}
    %\DeclareCaptionFont{blue}{\color{blue}} 

  %\captionsetup[lstlisting]{singlelinecheck=false, labelfont={blue}, textfont={blue}}
  \usepackage{caption}
\DeclareCaptionFont{white}{\color{white}}
\DeclareCaptionFormat{listing}{\colorbox[cmyk]{0.43, 0.35, 0.35,0.01}{\parbox{\textwidth}{\hspace{15pt}#1#2#3}}}
\captionsetup[lstlisting]{format=listing,labelfont=white,textfont=white, singlelinecheck=false, margin=0pt, font={bf,footnotesize}}
\renewcommand\lstlistingname{Codeblock}
 

\sloppy % Damit LaTeX nicht so viel über "overfull hbox" u.Ä. meckert

% Ränder
\addtolength{\topmargin}{-16mm}
\setlength{\oddsidemargin}{40mm}
\setlength{\evensidemargin}{40mm}
\addtolength{\oddsidemargin}{-1in}
\addtolength{\evensidemargin}{-1in}
\setlength{\textwidth}{13cm}
\addtolength{\textheight}{34mm}
%______________________________________________________________________

\hypersetup{
	pdfauthor = {Franziskus Domig, Stefan Gassner, BSc; BSc; Wolfgang Halbeisen, BSc; Matthias Schmid, BSc},
	pdftitle = {Projekt Dokumentation Roadrunner},
	pdfkeywords = {Roadrunner, Dokumentation, Transport, Logistik},
}

%======================================================================
%
%      Document
%
%======================================================================

\begin{document}

\selectlanguage{ngerman}

\pagestyle{empty} % Vorerst keine Seitenzahlen
\pagenumbering{alph} % Unsichtbare alphabetische Nummerierung

\begin{center}
\textsc{Fachhochschule Vorarlberg GmbH}\\

\vspace{5cm}
{\large\textbf{Projekt Dokumentation}}\vspace{.5cm}

{\LARGE Roadrunner}

\vspace{10cm}
ausgeführt von\\
{\large
Franziskus Domig, Bsc;\\
Stefan Gassner, BSc;\\
Wolfgang Halbeisen, BSc;\\
Matthias Schmid, BSc}\\
\vfill

\end{center}

\begin{tabular}{ll}
Bearbeitung: & Dornbirn, im Frühjahr 2011\\
Betreuer: & XXX\\
\end{tabular}
%______________________________________________________________________


\section{Vorwort}

TODO

\clearpage
\section{Spezifikation}
\label{sec:specification}

Transport von medizinischen Produkten (temperatursensitiv, Transportzeit, überwachbar, etc.).

\subsection{Technologie}

\begin{itemize}
	\item Backend-/Application-Server
	\item Annahmestelle für Waren (Item)
	\item verteilte Datenbank (CouchDB)
	\item Android für mobile Überwachung der Sensoren/Position/etc.
	\item Sensoren (Bluetooth, Wlan(?))
	\item Barcode-Lesegerät (evt. Android Devices)
\end{itemize}

\subsection{Fragestellungen}

CouchDB für diese Anwendung sinnvoll, machbar?

Barcode mit Android-Device lesbar? Schnell genug? Ist ein Framework verfügbar?

Sensoren mit Bluetooth auslesbar? PAN? Alternativen (evt. WLAN/HTTP)?



\clearpage
\section{Iteration 1}

TODO

\clearpage
\section{Sicherheit}

\subsection{Zeitsynchronisierung}

In diesem Abschnitt werden Probleme besprochen, die durch fehlerhafte respektive
mangelhaft durchdachte Zeitsynchronisierung oder Verbindungsabbruch entstehen
können.


\paragraph{Problem durch falsche Zeitstempel bei Logeinträgen:}
Betrachtet wird das Szenario ``Umladevorgang eines Produktes''. Das mobile
Gerät der Transporteinheit wird benutzt um den Ausladevorgang aus einem
Container im System zu verarbeiten. Mit dem scannen des Produkts wird auf dem
mobilen Gerät der Transporteinheit ein Logeintrag in dessen lokale Datenbank
erstellt. Genauso wird beim darauffolgenden Ladevorgang der Umladestation ein
Logeintrag auf dessen Gerät erstellt. Wenn das System mit absoluter Zeit
arbeitet und die Uhrzeit des Geräts der Transporteinheit vor jener der
Umladestation ist, dann würde im System der Übernahmevorgang der Umladestation
vor dem Ausladevorgang der Transporteinheit stattfinden.

Um dieses Problem zu lösen muss relative Zeit eingeführt und synchronisiert
werden. Für die Zeitsynchronisierung können bekannte Algorithmen für verteilte
Systeme eingeführt werden. Mögliche Algorithmen sind
 

TODO: UPV distributed Clocks .. algorithmen herausfinden und oben einfügen 
\paragraph

Grundsätzlich müssen diese Probleme berücksichtigt werden, in unserem Projekt
aus zeitlichen Gründen und anderer Zielsetzung nicht implementiert
wurden.


\clearpage
\section{Sensorik}

\subsection{Was für Sensoren werden verwendet?}


\subsection{Wann und wie kommt der Sensor ins System?}

\subsection{Wie wird er überwacht und erreicht?}

\subsection{Wie werden dem Mobilen Device Sensoren zugeordnet?}




\subsection{Simulation}

Wie sieht unsere Sensorsimulation aus? 

Was wird benötigt? 

Warum nodejs?


%______________________________________________________________________

\cleardoublepage

%\begin{thebibliography}{99}
%\addcontentsline{toc}{section}{Literaturverzeichnis}
%    Web-References
%______________________________________________________________________

%\hspace{-\leftmargin}{\Large\bfseries Web-Referenzen} % Wüster Hack %-|

%\end{thebibliography}


% ende des hauptteils
\fancyhead[R]{} % Keine Kopfzeile mehr oben auf jeder Seite

\end{document}
