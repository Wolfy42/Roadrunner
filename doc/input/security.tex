\section{Sicherheit}

\subsection{Zeitsynchronisierung}

In diesem Abschnitt werden Probleme besprochen, die durch fehlerhafte respektive
mangelhaft durchdachte Zeitsynchronisierung oder Verbindungsabbruch entstehen
können.


Problem durch falsche Zeitstempel bei Logeinträgen:
Betrachtet wird das Szenario ``Umladevorgang eines Produktes''. Das mobile
Gerät der Transporteinheit wird benutzt um den Ausladevorgang aus einem
Container im System zu verarbeiten. Mit dem scannen des Produkts wird auf dem
mobilen Gerät der Transporteinheit ein Logeintrag in dessen lokale Datenbank
erstellt. Genauso wird beim darauffolgenden Ladevorgang der Umladestation ein
Logeintrag auf dessen Gerät erstellt. Wenn das System mit absoluter Zeit
arbeitet und die Uhrzeit des Geräts der Transporteinheit vor jener der
Umladestation ist, dann würde im System der Übernahmevorgang der Umladestation
vor dem Ausladevorgang der Transporteinheit stattfinden.

Um dieses Problem zu lösen muss relative Zeit eingeführt und synchronisiert
werden. Für die Zeitsynchronisierung können bekannte Algorithmen für verteilte
Systeme eingeführt werden. Mögliche Algorithmen sind
 
\paragraph 
@TODO: UPV distributed Clocks .. algorithmen herausfinden und oben einfügen 
\paragraph

Grundsätzlich müssen diese Probleme berücksichtigt werden, in unserem Projekt
aus zeitlichen Gründen und anderer Zielsetzung nicht implementiert
wurden.
