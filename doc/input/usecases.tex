\section{Usecases}


\begin{itemize}
  \item Wareneingang
  	\subitem registriert Pakete im System
  	\subitem klebt QR-Code auf Pakete

	\item Logister plant und erstellt Lieferungen (neue Auftragsnummer wird
	generiert) \subitem wählt Pakete aus (können mit Sensoren bestückt sein)
		\subitem wählt Fahrer aus 
		\subitem wählt Transportmittel (können mit Sensoren bestückt sein) für
		Lieferungen aus
		\subitem trägt Zielort und Auftraggeber ein
  	
  \item Transporteur
  	\subitem holt oder hat Device mit Roadrunner App
  	\subitem loggt sich im Roadrunner System ein
  		\subsubitem mit Benutzerdaten wird sein/e aktuelle/r Auftrag/Lieferung aufs
  		Device synchronisiert ODER 
  		\subsubitem scannt Pakete und lädt sie in das vom Logistiker ausgwählte
  		Transportmittel
\end{itemize}

\paragraph{Daten-Synchronisierung}
	\textbf{Vorbedingungen: }
	\begin{itemize}
	  \item Transporteur hat sich in der System-App eingeloggt
	\end{itemize}
	
	Die Daten-Synchronisierung oder Replizierung wird durch das Einloggen im
	System angestoßen. Das mobile Gerät erhält folgende Information:
	\begin{itemize}
	  \item Adressen der Sensoren, die das Gerät überwachen sollte
	  \item alle Produkte, Pakete der aktuellen Lieferung, sowie Zielort, etc.
	  \item Überwachungs-Thresholds der Pakete
	\end{itemize}
\par
